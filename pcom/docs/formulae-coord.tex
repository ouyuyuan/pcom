
\subsection{Equations in Pressure-$\sigma$ Coordinate}

In order to model the general circulation in a compressible ocean,
our model will be formulated in a pressure coordinate system. Since
the surface pressure can vary with time and space, the normalized
pressure-$\sigma$ coordinate has therefore been adopted in this model,
which is defined as
\beeq \label{eq:def-sig}
\sigma=\frac{p-p_{t}}{p_{bt}},\,\, p_{bt}=p_{b}-p_{t},
\eneq
where $p_{b}$ is the bottom pressure and $p_{t}$ the atmospheric
pressure at the sea surface. $\sigma=0$ at sea surface and $\sigma=1$
at the bottom.

The horizontal momentum equations in the pressure-$\sigma$ coordinate can
be rewritten as (see appendix \ref{der:hor-mom-sig} for derivation)
\bese \label{eq:5}
\begin{align}
  \fd{u}{t} &= f^{*}v-\f{1}{a\cosf}\grp{ \fp{\phi}{\gl}+\f{1}{\rho}\fp{p}{\gl}
}+\md_u, \\
  \fd{v}{t} &=-f^{*}u-\f{1}{a}\grp{\fp{\phi}{\gf}+\f{1}{\rho}\fp{p}{\gf}}+\md_v,
\end{align}
\ense
The substantial time difference operator (see appendix
\ref{der:ddt-sig} for derivation)
\beeq \label{eq:21}
\fd{}{t}=\fp{}{t}+\f{u}{a\cosf}\fp{}{\gl}+\f{v}{a}\fp{}{\gf}+\dgs\fp{}{\gs}
\eneq
The viscous stress terms are parameterized as
(see appendix \ref{app:par-vis} for more information)
\bese \label{eq:37}
\begin{align} 
\mdu&=\rpbt\haru + \rpbt\prgapupps + 
  A_{m}\grp{\f{1-\tanfs}{a^2}u - \f{2\tanf}{a^2\cosf}\pvpl},\label{eq:24}\\
\mdv&=\rpbt\harv + \rpbt\prgapvpps + 
  A_{m}\grp{\f{1-\tanfs}{a^2}v + \f{2\tanf}{a^2\cosf}\pupl},
\end{align}
\ense
in which $A_m$ and $\gkm$ represent lateral and vertical viscosity, respectively.
$\gn_\gs$ is the horizontal gradient operator in pressure-$\gs$
coordinate. And 
\beeq
\haru = \f{1}{a^2\cosfs}\ppl\grp{\pbt A_m \pupl} +
\f{1}{a^2\cosf}\ppf\grp{\pbt A_m \cosf\pupf}
\eneq
Here we assume that lateral mixing of tracer is along the
pressure-$\gs$ surface. $\harv$ has a similar form as the above.

In order to form a set of finite difference schemes which may guarantee
the total energy be conserved, similar to \citet{Zeng1987}, we
introduce a new horizontal "velocity" $\vV$ 
\beeq
\vV=(U,V)=(\sqrt{\pbt}u,\sqrt{\pbt}v)
\eneq
Note that $\f{1}{2}\vV\cdot\vV$ just represents
the kinetic energy per unit volume in the pressure-$\sigma$ coordinates.
Multiplying by $\sqrt{p_{bt}}$ and with the aid of the mass conservation
equation, the momentum equations may be rewritten as 
(see appendix \ref{der:hor-mom2-sig})
\bese
\begin{align}
  \fp{U}{t}+\mm(U) &= f^{*}V - 
  \f{\sqrt{\pbt}}{a\cosf}\grp{\fp{\phi}{\gl}+\f{1}{\rho}\fp{p}{\gl}}
  +\sqrt{\pbt}\mdu \label{eq:MultiU}\\
  \fp{V}{t}+\mm(V) &= -f^{*}U - 
  \f{\sqrt{\pbt}}{a}\grp{\fp{\phi}{\gf}+\f{1}{\rho}\fp{p}{\gf}}
  +\sqrt{\pbt}\mdv \label{eq:MultiV}
\end{align}
\ense
 where $\mm$ is an advection operator
\beeq
\mm(\mu) = \racosf \left[\fp{(u\mu)}{\gl}-\f{\mu}{2}\fp{u}{\gl}+\fp{(v\mu\cosf)}{\gf}-\f{\mu}{2}\fp{(v\cosf)}{\gf}\right]
+\fp{(\dgs\mu)}{\gs}-\f{\mu}{2}\fp{\dgs}{\gs}
\eneq

The hydrostatic equation is
\beeq \label{eq:22}
  \fp{\gs}{z} = \fp{}{z} \grp{ \f{p-p_t}{\pbt} }
              = \f{1}{\pbt} \fp{p}{z}
              = -\f{\rho g}{\pbt}
\eneq
 
The mass conservation equation is (see appendix \ref{der:mass-sig} for derivation)
\beeq \label{eq:8}
\fp{\pbt}{t} + \gn \cdot (\pbt \vv) = 0,
\eneq
where $\vv = u\vel + v\vef + \dgs\ves$ is velocity vector, 
$\dot{\sigma}\equiv\frac{d\sigma}{dt}$.
and the transport term
\beeq \label{eq:11}
\gn \cdot (\pbt \vv) = \f{1}{a\cosf} \grpp{ \fp{(\pbt u)}{\gl} + 
\fp{(\pbt v\cosf)}{\gf} } + \fp{(\pbt \dgs)}{\gs}
\eneq
Equation \ref{eq:8} also serves as the surface pressure tendency equation.

The transport equations for tracers (potential temperature $T$ and salinity $S$)
are (see appendix \ref{der:tracer-sig} for derivation)
\bese
\begin{align} \label{eq:16}
  \pptpt+\divptv &= \pbt\mqt, \\
  \pptpt+\divpsv &= \pbt\mqs,
\end{align}
\ense
where $\divptv$ and $\divpsv$ have the similar form as Eq. \ref{eq:11}, and the
diffusion terms are parameterized as
\bese \label{eq:38}
\begin{align}
  \mqt &= \rpbt\hart + \f{1}{\pbt}\pps\grp{\f{\rho^2 g^2}{\pbt} \gkh\ptps},\\
  \mqs &= \rpbt\hars + \f{1}{\pbt}\pps\grp{\f{\rho^2 g^2}{\pbt} \gkh\psps},
\end{align}
\ense
in which $A_h$ and $\gkh$ represent lateral and vertical diffusivity,
respectively. Here we assume that lateral mixing of tracer is along the
pressure-$\gs$ surface.

Note that the tracers
equations have be rewritten in the flux form with the aid of the continuity
equation so that the prediction variables are $p_{bt}T$ and $p_{bt}S$
rather than $T$ and $S$ in order to keep the total tracers be conserved
in the numerical model.

Equations (\ref{eq:5}), (\ref{eq:8}), and (\ref{eq:16}) are prediction equations.

The state equation of seawater is
\beeq \label{eq:17}
\rho=\rho(T,S,p)
\eneq
 
The diagnostic equation of geopotential height is (see appendix
\ref{der:phi-sig} for derivation)
\beeq \label{eq:18}
\phi=gz_{b}+\int_{\gs}^{1}\f{\pbt}{\rho}d\gs
\eneq
 
The vertical velocity $\dgs$ is diagnosed by
(see appendix \ref{der:sig-w} for derivation)
\beeq \label{eq:19}
\pbt\dgs=\grp{\pbt\dgs}_{\gs=0}-\gs\fp{\pbt}{t}-
\racosf\int_{0}^{\gs}\grpp{ \fp{(\pbt u)}{\gl}+\fp{(\pbt v\cosf)}{\gf} }d\gs
\eneq

