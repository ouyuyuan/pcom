
\section{The Finite-Difference Scheme}

\begin{figure}[h!]
  \includegraphics[width=1.1\textwidth]{grid.eps} %
  \caption{\small PCOM horizontal stagger B-grid and vertical layer} \label{fig:grid}
\end{figure}

The grid system is a rectangular Arakawa staggered B-grid \citep{Bryan1969}
containing T cells and U cells (Fig.\ref{fig:grid}). 
The T cell is arranged at the integer
grid which defines the location of tracer quantities, and the U cell
is arranged at the half-integer grid which defines the location of
the zonal and meridional velocity components. 

In the vertical direction, there are also two types of grid. One is the grid
lies on the middle of the layers, the other is on the boundary of each layers.
The location of the vertical
velocity component $\dot{\sigma}$ is defined on the surface of T
cell (i.e., at the level $\sigma_{k-1/2}$), and the geopotential
$\phi$ is defined at the center of T cell (i.e., at the level $\sigma_{k}$).

In horizontal, we define T grid as integer grid, and in vertical, the middle of
layers as integer grid. We use $i,j,k$ to identify integer grid, use $ih, jh,
kh$ to identify the corresponding half-integer grid.

For example, we can classify the variables by grids:

\bd

\ii{ijk-grid} - $T$ (potential temperature), $S$ (salinity), 
$p$ (pressure), $w$ (vertical velocity), $\rho$ (water density), $\phi$
(geopotential height).
\ii{ihjhk-grid} - $U$ (zonal velocity), $V$ (meridional velocity).
\ii{ijkh-grid} - $p$ (pressure), $w$ (vertical velocity).
\ii{ij-grid} -  (sea surface height). 
\ii{ih-grid} -  \em{lat}
\ii{i-grid} -  \em{latt}

\ed

The finite-difference schemes are based on \citet{Zeng1987}.
Here we introduce the following notation for average and difference
at the half-integer grids and integer grids
\bese
\begin{align}
  \ol{F}{i+1/2} &= \f{F_{i+1} + F_i}{2},\\
      \ol{F}{i} &= \f{F_{i+1/2} + F_{i-1/2}}{2},\\
  \ul{F}{i+1/2} &= F_{i+1} - F_i,\\
      \ul{F}{i} &= F_{i+1/2} - F_{i-1/2},
\end{align}
\ense
And define
\bese
\begin{align}
  \ol{F}{k+1/2} &= \text{vertically interpolate integer level to half-integer},\\
      \ol{F}{k} &= \text{vertically interpolate half-integer level to integer},\\
  \ul{F}{k+1/2} &= F_{k+1} - F_k,\\
      \ul{F}{k} &= F_{k+1/2} - F_{k-1/2}
\end{align}
\ense

Thus, the surface pressure tendency equation, by integrating (\ref{eq:19}) in vertical
direction, may be written as (see appendix \ref{der:pre-tend} for derivation)
\begin{equation}\label{eq:33}
  \grpp{\pppt}_{i,j} = -\racosfjdl \uol{PU_B}{j}{i}  -\racosfjdf \uol{PV_B\cosf}{i}{j}
  -\rho_{0}gF_{E-P},
\end{equation}
where $P=\sqrt{\ol{\pbt}{i,j}}$, \index{$P$} $U_{B}=\int_{0}^{1}Ud\sigma$, 
and $V_{B}=\int_{0}^{1}Vd\sigma$.
This scheme guarantees total mass conservation if
\begin{equation}
\underset{i}{\sum}\underset{j}{\sum}F_{E-P}a^{2}\cos\gf_{j}\gD\gl\gD\gf=0
\end{equation}
 i.e., no net global or domain averaged freshwater flux cross the
air-sea interface.

Similarly, the diagnostic vertical velocity can also be obtained by integrating
(\ref{eq:19}) from surface to the $k$ layer of $\gs$, that is
\begin{equation}
[\pbt\dgs]_{i,j,k} = -\left[\rz g\fep + \gsk\pppt +
\racosfj \sumkzk \grp{ \uol{PU}{j}{i} + \uol{PV\cosf}{i}{j}}\gD\gsk\right]
\end{equation}

The momentum equations may be expressed by
\bese \label{eq:34}
\begin{align} 
  \ratihjhk{\pUpt} &= \grpp{ -\mm(U) + f^{*}V -\f{P}{\acosfjhdl}\grp{ 
\uol{\phi}{j}{i} + \avgij{\ga}\uol{P}{j}{i} } + P\mdu },\label{eq:39}\\
\ratihjhk{\pVpt} &= \grpp{ -\mm(V) - f^{*}U -\f{P}{\adf} \grp{ \uol{\phi}{i}{j} +
\avgij{\ga}\uol{P}{i}{j} }+ P\mdv },\\
\end{align}
\ense
in which $\alpha=\rho^{-1},\,\,  u=U/P,\,\, v=V/P$.  Note in PCOM $U,\, V$ are
the prognostic variables but the real velocity $u$ and $v$ are the diagnostic
variables.
The finite-difference scheme of the advection operator is 
 ( see \ref{app:1} for derivation )
 \beeq \label{eq:40}
\mm(\mu) = \f{1}{2\acosfjhdl} \left[ 2\ul{\ol{u}{i}\ol{\mu}{i}}{i} -
\mu\uol{u}{i}{i}\right] + \f{1}{2\acosfjhdf} \left[
  2\ul{\ol{\mu}{j}\ol{v\cosf}{j}}{j} - \mu\uol{v\cosf}{j}{j}\right] 
  + \rdsk\grp{\ul{\ol{\dgs}{i,j}\ol{\mu}{k}}{k} - \f{\mu}{2}\uol{\dgs}{i,j}{k}}
\eneq
where 
\bese
\begin{align}
  \gD\gsk &=\gs_{k+1/2} - \gs_{k-1/2},\\
  \gD\gskh &=\gs_{k+1} - \gs_{k}.
\end{align}
\ense

The viscous terms are
\bese \label{eq:42}
\begin{align}
  P\mdu &= \mathaa\mathab \notag\\
        &+\mathak\mathal - \matham\mathan\notag\\
        &+ \f{1}{P}\fdppk{\fdrsgsp \gkm\fdpups} 
        + A_m \grp{ \mathae U - \mathaf\mathag } \\
  P\mdv &= \mathaa\mathah \notag\\
        &+\mathak\mathao - \matham\mathap\notag\\
        &+\f{1}{P}\fdppk{\fdrsgsp \gkm\fdpvps}
        + A_m \grp{ \mathae V + \mathaf\mathaj }
\end{align}
\ense
The tracers equations are
\bese
\begin{align}
  \ratijk{\fp{(\pbt T)}{t}} &=-ADV(T)+DIF(T)\\
  \ratijk{\fp{(\pbt S)}{t}} &=-ADV(S)+DIF(S)
\end{align}
\ense
where
\begin{align} \label{eq:41}
  ADV(T) &= \racosfj \grp{ \fdppi{\avgj{PU}\avgi{T}} +
  \fdppj{\avgi{PV\cosfjh}\avgj{T}} } + \fdppk{\pbt \dgs\avgk{T}}\\
  DIF(T) &= \mathaq\mathar + \mathas\mathat - \mathau\mathav \notag\\
         &+\fdppk{\f{{\avgk{\rho}}^2 g^2}{\pbt}\gkv\fdppk{T}}
\end{align}
 It can be proven that the total salt is conserved if the natural
boundary conditions is applied. On the other hand, the total potential
temperature may be conserved if both the heat flux and fresh water
flux at sea surface are equal to zero, or, for the more general case
is, if
\begin{equation}
\underset{i}{\sum}\underset{j}{\sum}(F_h-C_p\rz \fep T_1)a^2\cosfj\gD\gl\gD\gf=0
\end{equation}
 Furthermore, it can prove that if all the variables are defined at
identical time levels and the frictional terms are neglected, the
above-given finite-difference equations can guarantee an accurate
conversion between the mechanical energy and internal energy.
