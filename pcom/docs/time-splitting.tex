
\section{Time Splitting Integration Method}

Since the model includes an explicit free surface, we therefore choose
to solve the full equations by means of a time-splitting integration
method (Zeng) to overcome the time step limit imposed by the gravity
wave processes. In other words, we may use a short time step to integrate
the geostrophic adjustment processes (mode-\mbox{I}) which involves
inertia external gravity wave, while apply a longer time step to the
quasi-geostrophic process (mode-\mbox{II}) which involves advection
and dissipation. Furthermore, a more longer time step is used to solve
the thermodynamic processes (mode-\mbox{III}).

(1) Mode-\mbox{I}: geostrophic adjustment process. It allows for the
momentum derivative associated with the pressure gradient and Coriolis
forces, as well as the surface pressure derivative associated with
the convergence or divergence of the mass flux, while the potential
temperature and the salinity are unchanged. 

Assume there is no water flux during the fast adjustment process, according to
(\ref{eq:33}), the prediction equation for surface pressure of \modo is
\begin{equation}
  \tnbpo{\pbt} = \tnbmof{\pbt} + 2\dtb \tnb{ -\racosfj \grp{\fdppi{\avgj{PU_B}} +
  \uol{PV_B\cosf}{i}{j}} }
\end{equation}

Neglect the advective terms $\mm(\mu)$ and the viscous terms $\md$ in the
horizontal momentum equations (\ref{eq:34}), we get the prediction equations for
horizontal velocity in \modo
\bese
\begin{align}
  \tnbpo{U} &= \tnbmof{U} + 2\dtb \grppp{ f^{*}\tavgnb{V} -\tnb{\f{P}{a\cosfjh} \grp{
  \ajpi{\phi} + \avgij{\ga}\ajpi{P} } }},\\
  \tnbpo{V} &= \tnbmof{V} + 2\dtb \grppp{ -f^{*}\tavgnb{U} -\tnb{\f{P}{a} \grp{
  \aipj{\phi} + \avgij{\ga}\aipj{P} } }},
\end{align}
\ense
where the superscript $n_{b}$ denotes the $n$-th time level of mode-\mbox{I};
$\gD t_{b}$ is the time step; the bar symbol represents temporal
filter \citep{Asselin1972} which would be used to remove the computational
mode associated with the standard leapfrog scheme.
Here a semi-implicit scheme is adopted to deal with the Coriolis terms.

From the definition of $\gs$ in (\ref{eq:def-sig}), we get the diagnostic equation
for pressure
\begin{equation}
[p]^{n_{b}}=\gs[p_{bt}]^{n_{b}}+[p_{t}]^{n_{t}},
\end{equation}
 where $n_{t}$ represents the current time levels of mode-\mbox{III}.

Since the calculation of the density is prohibitively expensive, on
the other hand, both the potential temperature and salinity are actually
held unchanged during the integration of mode-\mbox{I}, therefore,
instead of using the standard density formula, here the density is
taken the one calculated in $n_{t}$ time level by adding a perturbation
term due to change of the pressure. Thus the diagnostic equation of density is
\begin{equation}
[\rho]^{n_{b}}=[\rho(T,S,p)]^{n_{t}}+\left[\frac{\partial\rho}{\partial p}\right]^{n_{t}}([p]^{n_{b}}-[p]^{n_{t}})
\end{equation}

From (\ref{eq:18}), we have the diagnostic equation for geopotential height
\begin{equation}
[\phi]^{n_{b}}=gz_{b}+\int_{_{\gs}}^{1}\left[\frac{p_{bt}}{\rho}\right]^{n_{b}}d\gs
\end{equation}

(2) mode-\mbox{II}: advection and dissipation process. 
Only account the advective terms $\mm(\mu)$ and the viscous terms $\md$ in the
horizontal momentum equations (\ref{eq:34}), we get the prediction equations for
horizontal velocity in \modt
\bese \label{eq:43}
\begin{align}
  [U]^{n_{c}+1} &=U'+2\gD t_{c}\left([-\mm(U)]^{n_{c}}+P[\mdu]^{n_{c}-1}\right)\\
  [V]^{n_{c}+1} &=V'+2\gD t_{c}\left([-\mm(V)]^{n_{c}}+P[\mdv]^{n_{c}-1}\right)
\end{align}
\ense
where the superscript $n_{t}$ denotes the $n$-th time level of \modt;
$\gD t_{c}$ is the time step; $U'$ and $V'$ are the intermediate results of
$U$ and $V$ produced by integration of \modo at $(n_{c}+1)\gD t_{c}$ time levels.
The velocities used in the advection and dissipation operator are given by
\bese \label{eq:44}
\begin{align}
  [u]^{n_{c}} &=\left[\frac{U}{\sqrt{p_{bt}}}\right]^{n_{c}}\\
  [v]^{n_{c}} &=\left[\frac{V}{\sqrt{p_{bt}}}\right]^{n_{c}}
\end{align}
\ense

The diagnostic equation for vertical velocity is
(see appendix \ref{der:sig-w} for derivation)
\begin{equation}
[\pbt\dgs]^{n_{c}} = \racosf\int_{0}^{\gs} 
\left\{ \ppl\grpp{{\sqrt{p_{bt}}(U_{b}-U)}} + 
\ppf\grpp{\sqrt{p_{bt}}(V_{b}-V)\cosf}\right\}^{n_{c}}d\gs 
- (1-\gs)\rho_{0}gF_{E-P}
\end{equation}

 In addition, the Asselin temporal filter is applied to both $U$
and $V$ at $n_{c}\gD t_{c}$ time levels, i.e., 
\begin{equation}
[\mu]^{n_{c}}=\alpha_{t}[\mu]^{n_{c}}+(1-\alpha_{t})\left([\mu]^{n_{c}+1}+[\mu]^{n_{c}-1}\right)
\end{equation}
 where $\mu$ represents $U$ and $V$ , $\alpha_{t}$ is the temporal
filter coefficient.

The integrations of mode-\mbox{I} and mode-\mbox{II} are synchronous.
For example, when mode-\mbox{II} is integrated from $(n_{c}-1)\gD t_{c}$
time levels to $(n_{c}+1)\gD t_{c}$ time levels on a normal
leapfrog time step, mode-\mbox{I} will be therefore integrated $2\gD t_{c}/\gD t_{b}$
steps, of which the Euler backward scheme is applied to the first
time step and the leapfrog scheme applied to the rest.

(3)mode-\mbox{III}: thermodynamic process. The prediction equations
for the potential temperature and salinity may be written as
\bese
\begin{align}
  [p_{bt}T]^{n_{t}+1}&=[p_{bt}T]^{n_{t}-1}+2\gD
  t_{t}\left(-[\gn\cdot(\pbt T)]^{n_{t}}+[p_{bt}\mqt]^{n_{t}-1}\right)\\
  [p_{bt}S]^{n_{t}+1}&=[p_{bt}S]^{n_{t}-1}+2\gD
  t_{t}\left(-[\gn\cdot(\pbt S)]^{n_{t}}+[p_{bt}\mqs]^{n_{t}-1}\right)
\end{align}
\ense
 where $\gD t_{t}$ represents the time step for tracers. The
integrations of mode-\mbox{III} may be asynchronous with mode-\mbox{II}
if accelerated time steps are applied (Bryan, 1984).

Note that the prognostic variables of mode-\mbox{III} are $p_{bt}T$
and $p_{bt}S$ rather than $T$ and $S$ themselves. Correspondingly,
the advection terms in tracers equations are taken the flux form so
it is easy to keep the total tracers conserved.

The horizontal advection velocities for tracers may take the time
averaged velocities of mode-\mbox{I}, which can satisfy the continuity
equation (\ref{eq:8})
\begin{equation}
-2\gD t_{c}[\divpv]^{n_{t}}=[p_{bt}]^{n_{t}+1}-[p_{bt}]^{n_{t}-1}
\end{equation}

 In fact, if we set tracers be constant and discard the mixing terms,
the tracers equation will be just the continuity equation.

If the accelerated time steps are applied to the integration of tracers
equations, however, it will introduce a minor computational error
because $\gD t_{t}\neq\gD t_{c}$ thereby the continuity
equation (\ref{eq:8}) is no longer satisfied.

$T$ and $S$ may be calculated by 
\bese
\begin{align}
  [T]^{n_{t}+1} &=\frac{[p_{bt}T]^{n_{t}+1}}{[p_{bt}]^{n_{t}+1}}\\
  [S]^{n_{t}+1} &=\frac{[p_{bt}S]^{n_{t}+1}}{[p_{bt}]^{n_{t}+1}}
\end{align}
\ense
 And the density will be calculated according to the updated potential
temperature, salinity, as well as the pressure
\begin{equation}
[\rho]^{n_{t}+1}=\rho\left([T]^{n_{t}+1},[S]^{n_{t}+1},[p]^{n_{t}+1}\right)
\end{equation}
