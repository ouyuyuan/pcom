
\subsection{The Basic Equations}

The horizontal momentum equations in spherical coordinates may be
expressed in the form (see appendix \ref{der:mom} for derivation)

\bese \label{eq:hor-mom}
\begin{align}
  \fd{u}{t} &= f^{*}v-\f{1}{a\rho\cosf}\fp{p}{\gl}+\md_u, \\
  \fd{v}{t} &=-f^{*}u-\f{1}{a\rho}\fp{p}{\gf}+\md_v,
\end{align}
\ense
where $a$ is the radius of the Earth, $\gf$ latitude, $\gl$
longitude, $u$ and $ $$v$ are the zonal and meridional velocity components,
$f^{*}=2\Omega\sinf+u\tan\gf/a$ the apparent Coriolis
parameter, $\rho$ density, $p$ pressure, $\md_{u}$ and
$\md_{v}$ frictional force per unit mass. 

$d/dt$ is defines as (see \ref{der:diff} for derivation)
\beeq \label{eq:sub-diff2}
\fd{}{t}=\fp{}{t}+\f{u}{a\cosf}\fp{}{\gl}+\f{v}{a}\fp{}{\gf}+w\fp{}{z},
\eneq
where $w\equiv dz/dt$ is the vertical velocity. 


For the large-scale
motions, the vertical momentum equation is reduced to the hydrostatic
relation (see appendix \ref{der:mom} for derivation)
\beeq \label{eq:ver-mom}
\fp{p}{z}=-\rho g.
\eneq


The mass conservation equation is (see appendix \ref{der:mass} for derivation)
\beeq \label{eq:mass}
\f{1}{\rho}\fd{\rho}{t} + \divv = 0,
\eneq
where the velocity divergence $\divv$ is (see appendix \ref{der:mass} for derivation)
\beeq
\divv = \f{1}{a\cosf} \grpp{ \fp{u}{\gl} + \fp{(v\cosf)}{\gf} } + \fp{w}{z}.
\eneq

The equation of state of sea water can be expressed as
\beeq
\rho=\rho(T,S,p),
\eneq
where $T$ is the potential temperature, $S$ is salinity. 

The calculation
of the density is based on the formulas recommended by \citet{Millero1981}
by modifying the coefficients for inputting the potential temperature
\citep{Jackett1995}.

Note that we use the potential temperature as the tracer; thus,
during the invertible adiabatic processes, both the potential temperature
and salinity remain constant, The tracers equations are
\beeq \label{eq:12}
\frac{dT}{dt}=\mq_{T}
\eneq

\beeq
\frac{dS}{dt}=\mq_{S}
\eneq
 where $\mq_{T}$ and $\mq_{S}$ represent the turbulent viscosity and mixing.
