\section{Model grid}

PCOM use a stagger Arakawa B-grid (figure \ref{fig:grid}).

\begin{figure}[h!]
\centering
\includegraphics[width=\textwidth]{grid.eps}
\caption{PCOM horizontal stagger B-grid and vertical layer} \label{fig:grid}
\end{figure}

In horizontal projection, there are two types of grid, "T" grid and "U" grid.
U grid is where current components $u$ and $v$ lie on, and T grid is where 
temperature variables lie on.

In the vertical direction, there are also two types of grid. One is the grid
lies on the middle of the layers, the other is on the boundary of each layers.

In horizontal, we define T grid as integer grid, and in vertical, the middle of
layers as integer grid. We use $i,j,k$ to identify integer grid, use $ih, jh,
kh$ to identify the corresponding half-integer grid.

For example, we can classify the variables by grids:

\bd

\ii{ijk-grid} - $T$ (potential temperature), $S$ (salinity), 
$p$ (pressure), $w$ (vertical velocity), $\rho$ (water density), $\phi$
(geopotential height).
\ii{ihjhk-grid} - $U$ (zonal velocity), $V$ (meridional velocity).
\ii{ijkh-grid} - $p$ (pressure), $w$ (vertical velocity).
\ii{ij-grid} -  (sea surface height). 
\ii{ih-grid} -  \em{lat}
\ii{i-grid} -  \em{latt}

\ed

